\documentclass[conference]{IEEEtran}


\usepackage{cite}
\usepackage[colorlinks=false,bookmarks=false]{hyperref}
\usepackage{graphicx}
\usepackage{xcolor}
\usepackage{listings}
\usepackage{cite}
\usepackage{amsmath,amssymb,amsfonts}
\usepackage{cleveref}	% Needed for correctly referencing listings


% fatter TT font
\renewcommand*\ttdefault{txtt}


%%% ZF
\usepackage{listings}
\lstset{
	columns=fullflexible,
	%        basicstyle=\ttfamily\footnotesize,
	basicstyle=\ttfamily\small,      
	%columns=fullflexible, keepspaces=true,
	numbers=left,    
	numberblanklines=false,
	captionpos=b,
	%	breaklines=true,
	escapeinside={@}{@},
	numbersep=5pt,
	language=C,
	tabsize=2,
	breakatwhitespace=true,
	breaklines=true,
	deletekeywords={for},
	%        keywordstyle=\ttfamily
	numbersep=5pt,
	xleftmargin=.10in,
	%xrightmargin=.25in
}

\newcommand{\longlist}[3]{{\lstinputlisting[float, caption={#2}, label={#3}, frame=tb, captionpos=b]{#1}}}
\newcommand{\todo}[1]{\textcolor{red}{TODO: #1}}


\title{Constrained Random Assembly Test Program Generation for Processor Verification}

\author{
\IEEEauthorblockN{Tjark Petersen, Martin Schoeberl}
\IEEEauthorblockA{\textit{Department of Applied Mathematics and Computer Science} \\
\textit{Technical University of Denmark}\\
Lyngby, Denmark \\
s186083@student.dtu.dk, masca@dtu.dk}
}


\begin{document}

\IEEEoverridecommandlockouts

\maketitle

\pagestyle{empty}

\begin{abstract}

\end{abstract}

\begin{IEEEkeywords}
digital design, verification, Chisel, Scala
\end{IEEEkeywords}


\section{Introduction}
\label{sec:introduction}
Constrained randomized testing is a powerful tool to increase the efficiency of a verification engineer. 
One test case can capture a variety of different test values and each new test run provides greater 
confidence in the correct behavior of the feature of interest. In contrast, a simple directed test 
only shows correct behavior for some specific test values.

The methodology of constrained random testing has found its way into the verification of digital 
circuits, where the stimuli applied to a device under test are generated randomly following a set 
constraints. When it comes to the verification of the digital circuit of a processor this method 
is not suitable though, since the element determining most of the circuit behavior is the memory 
which contains the actual program run on the processor. Here a framework is needed which to some 
extend knows about the targeted instruction set and the randomizable portions of an individual 
instruction. This framework can be extended to provide a full processor verification suite which 
automatically generates instruction sequences for the purpose of verifying the correct behavior 
of a processor implementation.

\section{Related Work}

Many solutions to the automation of test generation have been proposed, \todo{commercial products, riscv-torture and riscv-dv}

% https://research.ibm.com/haifa/dept/vst/tgt-activities.shtml
% https://github.com/ucb-bar/riscv-torture
% https://github.com/google/riscv-dv

Those often come inside of a big verification framework. This projects aims at providing a 
small and easy to use interface inside of Scala which can be used for quick unit test like 
programs which can be used to test Chisel processor implementations.

These frameworks are relatively heavy to deploy and enforce their own way of co-simulation (???).


In this paper a more lightweight framework is proposed which only focuses on the test generation 
for different instruction sets. It aims to provide an easy to use and randomized alternative to 
directed assembly test programs used for the basic verification of a processor. The low-overhead 
generation process of the proposed framework allows it to be used in a unit test like manner, 
where many isolated features can be tested in their own test case, but where the specific 
instructions are randomized within the bounds of the given constrains.


\section{The generation framework}


The problem of constraint random test program generation for different instruction sets comes with 
two key sub-problems. On the one hand a general language model for the different potential target 
languages needs to be formulated. On the other hand, a decision has to be made concerning which 
parameters and traits of the generated program sequence should be constrainable and in what form.

Due to the inherent low level of abstraction in assembly code, a model capturing different target 
instruction sets can be quite simple. An instruction set consists of different instructions having 
their own mnemonic, operands, syntax and semantics. The operands can be randomized inside their 
respective domains.

When it comes to the constraining of a program sequence, two levels can be identified: instruction 
level constraints and program level constraints. The former applies to randomizable operands inside 
of an instruction whose domains can be further limited by constraints. The latter concerns potential 
relational constraints between adjacent instructions in a program or the overall distribution of 
instructions with respect to a parameter like for instance a specific category they fit into. 
Furthermore, a mixture between fixed instructions and randomly selected instructions in a program 
sequence could be a desired form of constraining a program.

To tackle the this problem, a construct named \textit{Pattern} is introduced. A pattern contains 
a sequence of instructions, where some of them are specific instructions with potentially specific 
operands whereas others portions of the pattern are left to be filled in randomly. Some of the 
items in the pattern sequence can be a pattern themselves, creating a nested system of patterns. 
This resembles a tree with nodes being patterns or instructions and where actual instructions are 
the leafs. In order to generate a program sequence from a pattern, it invokes a \texttt{produce} 
method on all its sequence items. This creates a call chain down through the tree until an instruction 
is met which just produces a copy of itself.


In order for a potential user to set only some operands of an instruction when describing a pattern, 
an interface is needed. Furthermore, an instruction needs to remember whether and which operands 
have been set, so that a randomized copy maintains the same operand value. This is achieved using 
the Scala \texttt{Option} type. Instructions have optional parameters for their operands which by 
default are set to \texttt{None}. At class initialization for each operand either a random value 
inside the operand domain or a passed initializer different from \texttt{None} is chosen. An example 
of this behavior can be seen in listing \todo{fix} where the immediate operand of the Risc-V 
load immediate instruction can be left blank and filled in randomly.


\begin{lstlisting}{scala}
val forLoop = Pattern(implicit c => Seq(
 LI(t1,0),
 LI(t2),
 Label("ForLoopStart"),
 Instruction.fillWithCategory(10)
                  (Category.Arithmetic),
 ADDI(t1,t1,1),
 BLT(t1,t2,LabelRecord("ForLoopStart")),
 Instruction.fill(10)
))
\end{lstlisting}


In listing \todo{fix} it can be seen that the instruction sequence is actually defined 
inside an anonymous function which takes an argument \texttt{c} that is declared implicit in the 
function context. This argument is the \texttt{GeneratorContext} which is passed down implicitly 
through all \texttt{produce} calls in order to give access to mutable state like for instance the 
current program counter, as well as access to functions which produce new random instructions.

% the generation user interface

% the generation context

% patterns and instructions 
% patterns -> code reuse

% isa definitions

% Label insertion

As a way to control the instruction stream from a higher level of abstraction, a category system is introduced.

A test generator can be instantiated given an instruction set and a set of constraints. The constraint 
system is easily extendable and includes for now interfaces to constrain the following parameters:
\begin{itemize}
    \item category distribution
    \item 
\end{itemize}


\section{Evaluation}

% non-architectural specific -> might be harder to capture some behavior but Patterns are quite flexible





\section{Conclusion}
The framework could be extended to use fuzzing techniques to automatically generate interesting test using a generation-metric collection-mutation loop. For this a mutation interface would have to be added to the framework.

\subsection*{Source Access}

This work is in open source and hosted at GitHub:\\ \url{https://github.com/chiselverify/chiselverify}.
We plan also to regularly publish it on Maven.\footnote{https://mvnrepository.com/artifact/io.github.chiselverify/chiselverify}

\subsection*{Acknowledgment}


\bibliographystyle{IEEEtran}
%\bibliography{../msbib,../chisel-uvm}%,../funding/ftp-chisel/testing}


\end{document}
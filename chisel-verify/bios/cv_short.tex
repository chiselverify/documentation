\documentclass[%draft,
    a4paper,
    12pt, % use explicit paper size
    headinclude, footexclude,
    notitlepage,
    headsepline,
    pointlessnumbers,
    ]{scrartcl}
\usepackage{pslatex} % -- times instead of computer modern

\typearea{16}
\usepackage{scrpage2} % for headers
 \setkomafont{pagehead}{\scshape\small}
 \setkomafont{pagenumber}{\scshape\small}
 \ihead[]{Schoeberl}
 \chead[]{}
 \ohead[]{\projname}

% \ofoot[]{} \cfoot[]{} \ifoot[]{}

\usepackage{hyperref}
\usepackage{booktabs}
\usepackage{graphicx}
\usepackage{amsmath}
\usepackage{dcolumn}
\newcommand{\cc}[1]{\multicolumn{1}{c}{#1}}
\newcolumntype{d}[1]{D{.}{.}{#1}}
\usepackage{boxedminipage}
\usepackage{pdfpages}
\usepackage{xspace}

\newcommand{\code}[1]{{\textsf{#1}}}
\newcommand{\todo}[1]{{\emph{TODO: #1}}}


\begin{document}

%\pagestyle{scrheadings}


\section*{Curriculum Vitae Martin Schoeberl}

\noindent DI Dr.~Martin Schoeberl\\
Mariendalsvej 25, 1\\
DK-2000 Frederiksberg, Denmark\\
\url{mailto:masca@dtu.dk}\\
\url{http://www2.imm.dtu.dk/people/masca}

%Martin Schoeberl is associate professor at the Technical University of Denmark, at
%the Department of Informatics and Mathematical Modelling (DTU Informatics).
%He completed his PhD at the Vienna
%University of Technology in 2005 and the Habilitation in 2010.
%%
%Martin Schoeberl's research focus is on time-predictable computer architectures and
%on Java for hard real-time systems. During his PhD, he developed the
%time-predictable Java processor JOP. This processor is currently
%being used in industrial projects and is the basis for further
%research on chip-multiprocessors for real-time systems. JOP is the only Java processor for
%which the execution time can be easily analyzed and worst-case
%execution time (WCET) analysis tools are available.
%%
%According to Microsoft Academic Search he is ranked on place 8 for the Top authors in Real-Time \& Embedded Systems in the last 5 years.\footnote{Last accessed on 13.2.2012:
%\url{http://academic.research.microsoft.com/RankList?entitytype=2&topdomainid=2&subdomainid=19&last=5}}
%%
%Martin Schoeberl was the architecture work package lead
%in the EC funded project JEOPARD. He was leading the proposal and
%is now technical lead of the EC funded project
%T-CREST (Time-predictable Multi-Core Architecture for Embedded
%Systems), started September 2011.

\subsubsection*{Education}

\begin{tabular}{rl}
December 2010 & Habilitation at TU Vienna \\
April 2005   & PhD Degree in Computer Engineering, with distinction from TU Vienna\\
November 1994 & Master's Degree in Computer Science from TU Vienna\\
1980 -- 1986 & Engineering School for Communications Engineering\\
             & and Electronics in St.\ P\"olten\\
\end{tabular}

\subsubsection*{Employment}
\begin{tabular}{rl}
Since 2010    & Associate Professor at DTU Compute\\
2005 -- 2009 & Assistant Professor at the Institute of Computer Engineering, TU Vienna\\
%1996         & Civilian Service in Vienna\\
Since 1994   & Self-employed with projects in automation and supervision\\
1992 -- 1994 & Software engineer at Wirtschafts- und Sozialwissenschaftliches Rechenzentrum\\
1987 -- 1991 & Software engineer at COIN Computerentwicklungen GmbH\\
1986 -- 1987 & Software engineer at SYSGRAPH Computergraphik GmbH\\
\end{tabular}


\subsubsection*{Summary of Scientific Work}

I have published two books, 16 journal articles, one book chapter, one patent, and over 100
papers in peer reviewed conferences and workshops, 16 invited talks.  My h-index is 28.
%According to Microsoft Academic Search I am ranked on place 2 for the Top authors in Real-Time \& Embedded Systems in the last 5 years.\footnote{Last accessed on 20.10.2014:
%\url{http://academic.research.microsoft.com/RankList?entitytype=2&topdomainid=2&subdomainid=19&last=5}}

I am reviewer for several journals and PC member of several conferences and
member of the EG for the Safety Critical Java Specification (JSR 302).
I was program chair for JTRES 2009 and JTRES 2016, WCET 2016; co-chair for RTS ACM SAC 2011;
and  guest editor for the special issue on \emph{Java
      Technologies for Real-Time Distributed and Embedded
      Systems} in Concurrency and Computation: Practice and
      Experience.
I served as program chair for ISORC 2015.


The result of my PhD thesis, the real-time Java processor JOP,
enabled my participation in the EU project JEOPARD (Java Environment
for Parallel Realtime
Development).
I was responsible for the Austrian part of the proposal, representing the
TU Vienna in the project, and I lead the ``Architecture"
work-package in JEOPARD.
My work on time-predictable architectures led to the EC funded project
T-CREST (Time-predictable Multi-Core Architecture for Embedded
Systems). T-CREST started September 2011 and the overall project budget
is EUR 3.8 million. I led the funding proposal and was technical lead of
T-CREST.

I've supervised 2 PhD students in Vienna (finished 2011 and 2012), two PhD
students at DTU (finished 2014 and 2016) and am currently co-supervising at DTU
one PhD student.

%\end {document}

\subsubsection*{Research Visits}

\begin{itemize}
  \item February (+ spring part time) 2006, CBS, Copenhagen,
      Denmark
  \item August 2007, Aalborg University, Denmark
  \item April 2008, Aalborg University, Denmark
  \item Fall 2009 and fall/winter 2012, University of California, Berkeley
\end{itemize}

\subsubsection*{Community Services}

\begin{itemize}
  \item General program chair JTRES 2009, track co-chair ACM SAC 2011, program chair ISORC 2015
  \item Workshop chair JTRES 2007 and JTRES 2012
  \item Guest editor for the special issue on \emph{Java
      Technologies for Real-Time Distributed and Embedded
      Systems} in Concurrency and Computation: Practice and
      Experience
  \item Expert Group member of the Safety Critical Java
      Technology Specification (JSR 302)
  \item Editorial Board Member of Journal of Systems Architecture (JSA),
  associate editor: EURASIP JES, IJERTCS
  \item PC member: FPL 2007--2014, JTRES 2006--2014, ISORC 2010, 2012--2015, ACM SAC 2010--2014, SEUS 2010, 2013, Transact 2010, RTNS 2012, WCET 2012
  \item Reviewer for embedded systems journals: RTS, JSA, EURASIP JES,
      TECS, TCAD, TII, TPDS, TCAS, SP\&E, IET Software, TPDS, S P\&E, IJERTCS, ESL, TAES
\end{itemize}







\subsubsection*{Funding ID}

\begin{itemize}
  \item 2004-2007: \emph{Principal Investigator} and funding
      applicant (self employed) for the national SME funding
      project: Implementation of the CLDC standard for real-time
      systems on a Java processor, EUR 80,000.-

  \item 2008-2010: FP7 EC project JEOPARD (Java Environment for
      Parallel Realtime Development) under grant agreement number
      216682; EUR 3,170,000.- (total), EUR 167,000.- for the TU
      Vienna. I wrote the proposal part for TU Vienna and 
      led the work package on architectures. 
      
   \item 2011-2014: Certifiable Java for Embedded Systems (Funding applicant \& PI),
   Danish Research Council for Technology and Production
   Sciences under contract 10-083159; DKK 5,058,000.-.
     
  \item 2011-2014: FP7 EC project T-CREST (Time-predictable
  Multi-Core Architecture for Embedded Systems) under grant
  agreement number 288008; EUR 3,807,000.- (total), EUR 703,000.- for DTU.
  I \textbf{led the funding proposal} and am now \textbf{technical coordinator of T-CREST}.
  The proposal was \textbf{ranked at position 4 out of 59} submissions.
  
  
  \item 2013-2015: Hard Real-Time Embedded Multiprocessor Platform - RTEMP
  (Co-applicant, PI is Jens Spars{\o}), 
  Danish Research Council for Technology and Production
  Sciences under contract nn-nnnnnn; DKK 4,977,862.-
\end{itemize}



\subsection*{Publications}

I have published two books, 16 journal articles, one book chapter, one patent, and over 100
papers in peer reviewed conferences and workshops. Following lists the five most important
publications.

\begin{enumerate}

\item Martin Schoeberl, Benedikt Huber, and Wolfgang Puffitsch.
 Data cache organization for accurate timing analysis.
 {\em Real-Time Systems}, DOI: 10.1007/s11241-012-9159-8:1--28, 2012.

\item Martin Schoeberl.
Scheduling of hard real-time garbage collection.
{\em Real-Time Systems}, 45(3):176--213, 2010.

\item Martin Schoeberl, Wolfgang Puffitsch, Rasmus~Ulslev
    Pedersen, and Benedikt Huber. Worst-case execution time
  analysis for a {Java} processor. {\em Software: Practice and
  Experience}, 40/6:507--542, 2010.

\item Martin Schoeberl. Time-predictable computer architecture.
    {\em EURASIP Journal on Embedded Systems}, vol. 2009, Article
    ID 758480:17 pages, 2009.
    
\item Martin Schoeberl. A Java processor architecture for
    embedded real-time systems. {\em Journal of Systems
    Architecture}, 54/1--2:265--286, 2008.

\end{enumerate}



\end{document}

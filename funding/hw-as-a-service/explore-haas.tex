

\documentclass[fleqn,12pt]{article}
\usepackage[a4paper,top=2cm,bottom=2cm,left=2cm,right=2cm]{geometry}
\usepackage{times}
\usepackage[danish,english]{babel}
\usepackage[utf8]{inputenc}
\usepackage[T1]{fontenc}
\usepackage{microtype}
% \usepackage{graphicx}         % For PDF figures
% \usepackage[dvips]{graphicx}  % For EPS figures, using dvips + ps2pdf

\usepackage[colorlinks=true,linkcolor=black,citecolor=black]{hyperref}
\usepackage{booktabs}

\usepackage{tikz}
\usetikzlibrary{positioning,fit}
\usetikzlibrary{shapes,backgrounds}
\usetikzlibrary{arrows,fit,automata,positioning,decorations,calc}
\usetikzlibrary{spy}
\usetikzlibrary{matrix,chains,decorations.pathreplacing}
\usepackage{pgfgantt}

\newcommand{\code}[1]{{\textsf{#1}}}

% Adding comments in the text during writing process
\newcommand{\todo}[1]{{\it TODO: #1}}
\newcommand{\note}[1]{{\it Note: #1}}
\newcommand{\martin}[1]{{\color{blue} Martin: #1}}
\newcommand{\jens}[1]{{\color{green} Jens: #1}}

% uncomment following for final submission
%\renewcommand{\todo}[1]{}
%\renewcommand{\note}[1]{}
%\renewcommand{\martin}[1]{}
%\renewcommand{\jens}[1]{}




\usepackage{listings}
\lstset{
	columns=fullflexible,
	basicstyle=\ttfamily\small,
	numbers=left,
	numberblanklines=false,
	captionpos=b,
	escapeinside={@}{@},
	numbersep=5pt,
	language=C,
	tabsize=2,
	breakatwhitespace=true,
	breaklines=true,
	deletekeywords={for},
	numbersep=5pt,
	xleftmargin=.10in,
}



\begin{document}

\begin{center}
  {\LARGE\bf Exploring Hardware as a Service (ExHaaS)}\\[1ex]
  {\large DIREC Explore Project}\\[1ex]
  {\large Martin Schoeberl, DTU Compute}\\[1ex]
 \end{center}


%\section{Abstract for the application -- do not include in this document}
%
%Performance increase with general-purpose processors has come to a halt. We can no longer depend on Moore's Law to increase computing performance. The only way to achieve higher performance or lower energy consumption is by building domain-specific hardware accelerators. These accelerators can be built in ASICs or in FPGAs in the cloud. To efficiently design and verify those domain-specific accelerators, we need agile hardware development.
%
%
%This project aims to develop a method and concrete tools for agile hardware development. We will use tools, languages, development, and testing methods from the last decades in software development and apply them to hardware design. We aim to raise the tooling level for a digital design to increase productivity. Time for verifying (testing) of digital systems is about double the time of developing them in the first place. Therefore, this project's central focus is on applying software development testing methods to hardware development.
%
%\subsection{Popular Abstract}
%
%Digital systems are a central part of our current and future digital enhanced live. Denmark has a considerable industry in the design and development of digital systems. However, designing and verifying such digital systems becomes an ever-growing challenge. The main issue is the usage of old tools compared to tools for programming those devices. This project aims to apply tools and methods from software development to hardware development to increase productivity. Danish firms will benefit from the results of this project.
\begin{abstract}
\noindent
Digital systems are a central part of our current and future digital enhanced live. Denmark has a considerable industry in the design and development of digital systems.
However, designing and verifying such digital systems becomes an ever-growing challenge. The project addresses: (1) how hardware verification can learn from software testing, (2) co-verification of software and hardware with modern tooling, and (3) how to simplify software developers' access to hardware accelerators in the cloud.
%This project requests funding for two PhD students at DTU and ITU.
\end{abstract}

\subsection*{Project type: Explore}

\subsection*{Project period: start 4/2022, 1 year}

\section*{Participants \& Collaborators}

\noindent DTU: Martin Schoeberl and Jan Madsen

\noindent ITU: Mahsa Varshosaz and Andrzej Wasowski

\paragraph*{Human Resources}

For the HaaS project, we request funding of one part time PostDoc and student researchers.
Each of the senior researchers will contribute to the HaaS research project.
Furthermore, several students will contribute to the project with
master projects.
%
%We intend to build a group with one PhD student, one postdoc, and
%two senior researchers at DTU.
%
%Quoted from the Diversity and Gender statement at DTU:
%``Diversity, equal treatment, and equality are integral to DTU, being an international
%university in scope and standard, and are fundamental principles underlying DTU's
%expectations of respect and equality''.
%As the already named researchers are all male, we will actively search
%for female researchers for the PhD positions.
%However, the PhD position will be announced openly and men and women
%will have equal opportunities for applying.


{\bf Martin Schoeberl (MS)} is associate professor at DTU Compute and is the PI.
%His research interest is in computer architecture for real-time systems.
During his stay
at UCB in 2012 he picked up Chisel and brought it to DTU in research and teaching.
Martin has {\bf written the Chisel textbook}~\cite{chisel:book}, which has been translated
into Chinese, Japanese, and Vietnamise. Martin is a {\bf member of the Technical Advisory Committee for
Chisel} and therefore keeps the work of HaaS in sync with the Chisel main development.
Martin is part of the regular Chisel developer meeting with
UC Berkeley researchers and developers from SiFive, a startup in silicon valley.


{\bf Jan Madsen (JM)} is professor and deputy director of DTU Compute.
His research spans methods
and tools for systems engineering of computing systems, embedded systems-on-a-chip,
Cyber-Physical Systems (Internet-of-Things), microfluidic biochips (Lab-on-Chip), and
synthetic biology (molecular computing).

%{\bf Zhoulai Fu (ZF)} is an assistant professor at ITU. His research
%spans programming language theory and software engineering
%techniques. Related to this project, he worked on developing automated
%testing techniques in achieving high code coverage in scientific
%programs and detecting bugs in robotic software.

{\bf Mahsa Varshosaz (MV)} is an assistant professor at ITU (starting 2022). Her research involves developing and application of testing and verification techniques. In the past few years, she has worked on developing modeling and model-based testing approaches for highly configurable systems. She has collaborated with researchers and engineers in ASML Company and TNO (Netherlands Organisation for Applied Scientific Research) to provide domain specific languages for describing specification of real systems and to bring better structure and rigour to the applied testing process.

{\bf Andrzej Wasowski (AW)} is professor at ITU. He heads the Software Quality research group.  He works on increasing software quality by building software development tools. AW contributed to formal modeling, semantic analyzers for software (including the Linux kernel and Robot Operating System), software testing, transformation, modernization, and design of domain specific languages.
\looseness=-1

%{\bf Peter Sestoft (PS)} is professor and head of department at the IT University of Copenhagen.
%His research focus is programming languages, and especially functional, parallel, domain-specific
%and declarative languages, their description, formal modeling, analysis, transformation, and implementation.
%Most of his books have appeared with leading international academic publishers.

{\bf Torur Biskopsto Strom} got his PhD from DTU Compute. His work was on hardware support for multicore synchronization.
Therefore, he is very well educated to explore hardware as a service. We plan to hire him part time.

The topic is relevant to several companies in Denmark. Following companies are committed to contribute to the project: 
Comcores (Thomas Noergaard, Joergen Carstensen) and Microchip (Thomas Aakjer)

%Napatech (Jesper Birch), Comcores (Thomas Noergaard, Joergen Carstensen), Microchip (Thomas Aakjer), and SyoSil (Jacob Sander Andersen).


We are in contact with following additional companies that have shown interest in
the project (e.g., during the initial expression of interest).
They may join the project later in one form or another:
Teledyne Reson (Morten Rytter, Simon Andersen), WSAudiology (Ketil Julsgaard),
Intel (Dines Justesen), GN Hearing (Mark Brooks), and Demant (Anders Hebsgaard).


Comcores is a key player in the field of digital IP solutions for communications with a particular focus on time sensitive networks, switching architectures and 5G fronthaul and radio.
Comcores is interested in improving the throughput with HaaS.
Comcores is very keen on taking an active role as an industrial partner in the DIREC project, since it clearly maps to our specific expertise and areas of interest.

Microchip is interested in using Chisel to improve the design throughput and
provides a hardware use case to sort Ethernet packets according to their deadlines
for a time-sensitive network switch.
%
%Syosil will provide consulting and training in connection with any questions related
%to functional verification and UVM.
%
%Teledyne is interested in exploring Chisel for development and testing with the compatibility of VHDL and Verilog.
%%
%WSAudiology provides a decimation filter as a use case for our verification framework.
%%


%\subsection*{Sent out reminders:}
%
%Jacob (Syosil), Morten (Teledyn), Thomas (Microchip)


\subsection*{Workstreams: WS 6 and WS 7}


\section*{Contact Details}

Martin Schoeberl

\noindent Associate Professor

\bigskip

\noindent DTU Compute

\noindent Department of Applied Mathematics and Computer Science

\noindent Technical University of Denmark

\noindent Richard Petersens Plads

\noindent Building 322, room 128

\noindent 2800 Lyngby

\noindent Denmark

\bigskip


\noindent Phone: +45 50621247

\noindent Email: masca@dtu.dk


\newpage
\section*{Project Description}


This project aims to develop a method and concrete tools for agile hardware development.
We will use tools, languages, development, and testing methods from the last decades in
software development and apply them to hardware design.
We aim to {\bf raise the tooling level for a digital design to increase productivity}.
Time for verifying (testing) of digital systems is about double the time of developing
them in the first place.
Therefore, this project's central focus is on {\bf applying software
testing methods for hardware testing}.

We will build a combination of open-source tools for verifying
circuits described in mixed languages (VHDL, SystemVerilog, and Chisel).
It builds on top of the Chisel hardware construction language and uses Scala to drive the verification.
%We will explore the testing strategy used in UVM in the context of verifying hardware described in Chisel.
We have in an earlier, small project explored initial steps towards
the project aim \cite{verify:chisel:2020}.


%This project proposes a research project that aims at building a testing framework
%in Scala that takes the best methods from UVM and from decades of experience
%in software testing.
%The developed framework shall support mixed languages (VHDL, SystemVerilog, and Chisel)
%to be able to integrate legacy code.
%Furthermore, our aim is to build on open-source projects. Therefore, our
%work will be in open-source as well.

%\begin{itemize}
%\item Maybe more ideas: \url{https://www.youtube.com/watch?v=dbOi_Gboi_0}, \url{https://www.youtube.com/watch?v=4FCZLrauDcE}
%\item Higher-Order Hardware Design, meta-programming language and the actual hardware construction language are the same, usually Python or Perl scripts with strings
%\item Have a measurable objective (LoC UVM vs Scala, SystemVerilog vs Chisel)
%\item Chisel has all the Scala/Java tooling and libraries available, SystemVerilog is a niche language
%\item Rise level of tooling, not necessarily level of abstraction in HW description
%\item IDE
%\item Namespace with packets make it easier to combine IPs
%\item Industry issue is verification: how from Chisel to VHDL/Verilog
%\item How much ASIC design is done in DK? Revenue numbers?
%\item There are not enough HW designers and verification engineers available, so they shall be more productive
%\end{itemize}



As prerequisites, all researchers involved in this project will need to learn
languages and tools involved in the project and related work.
They will learn about the Scala programming language and ScalaCheck
(a Scala implementation of the property-based testing) on the software side.
On the hardware side,
the researchers need to get familiar with Chisel and SystemVerilog.


\paragraph{Hardware Generators.}

The productivity of hardware design can be greatly increased by developing so-called
hardware generators. A hardware generator is a program that can generate a configurable
hardware description.
Scala with functional programming is an excellent basis for developing a methodology for the
development of such hardware generators.


%\subsection{WP2}
\paragraph{Property-based Testing.}
As a starting point, we will use ScalaCheck, an implementation of
property-based testing in Scala, for hardware testing. We will invite
hardware developers to write properties as constraints (like
propSquare above), and then we will use ScalaCheck to validate or
refute those constraints. Property-based testing has seen notable
successes previously, such as in locating a long-standing concurrency
bug in the Erlang database server~\cite{DBLP:conf/erlang/HughesB11};
it was also used by Ericsson to test its media
proxy~\cite{DBLP:conf/erlang/ArtsHJW06}, by Volvo to test car
communications protocols~\cite{DBLP:conf/icst/ArtsHNS15}.


%\paragraph{Co-simulation.}
%
%\todo{A figure would be nice.}
%
%\begin{itemize}
%\item One use case for evaluation: cosimulation of a RISC-V simulator (Tommy) with an OS RISC-V HW
%\item Ptolemy~\cite{ptolemyII-book} can be used to co-simulate the environment, supporting a model based design.
%\item Integration of C/C++ based models in the verification with Scala
%\item Java/Scala in UVM
%\item Hw/sw co-verification with Scala and so on, e.g., run an application on a SW processor model exploring some hardware artifacts (could be S4NOC)
%\item Model based design (Jan) with co-simulation
%\end{itemize}
%
%\paragraph{Assertions.}
%
%\begin{itemize}
%\item assertions during simulation
%\item Assertion ave been long part in SW, begin of C, but seldom used in HW and more complex assertions are interesting, such as when req is asserted, an ack has to become active within 5 clock cycles
%\end{itemize}

\paragraph{Constraint Random Testing and Fuzzing Techniques.}

Constraint random testing in hardware is similar to fuzzing in software testing.
We will add support for constrain random testing to ChiselTest~\cite{chisel:tester2}.
Once we have tools to instrument the hardware under test with a user-written
specification or automatically generated specification, we will be able to use fuzzing techniques
to validate those specifications. To this end, we will work
with our industrial collaborators to get access use cases serving
as the DUT.

\paragraph{Test Coverage.}

Code coverage is a useful tool for verifying digital designs
since it allows one to see which parts of their design have actually been tested correctly.
We will implement coverage inside of the execution engine of the Chisel simulator using a technique
presented by Ira. D. Baxter~\cite{branch-cov-made-easy:2002}.
We will add a method to specify \textit{functional coverage points}, also known as
\textit{coverage groups} in SystemVerilog.


%Treadle, which is a simulation engine for Chisel, does not contain support for measuring code coverage.
%We will added branch coverage to Treadle so that one can see which lines of
%LoFIRRTL code, the intermediate representation used inside of Treadle, were covered by a series of tests
%and then know, using that information, which multiplexer paths were tested.
%Furthermore, we plan to map the results obtained with LoFIRRTL code back to the source Chisel description.
%This mapping can be done using treadle's internal ``source trackers'' that associate some FIRRTL lines back
%to their chisel source. Once that is done, it will be interesting to move from branch coverage to functional
%coverage, which would require a way to define \textit{functional coverage points} also known as
%\textit{coverage groups} in SystemVerilog.

%\begin{itemize}
%\item On coverage and cover points in proposal (coverage of the RTL hardware, but also on the generator (Jack's comment, see \url{https://gitter.im/freechipsproject/chisel3?at=5f63c878603d0b37f43b67f3})
%\item Also have range coverage on individual signals and the matrix of 2 or more (see UVM example).
%\end{itemize}
%


\paragraph{Verification Framework.}

We will develop an object-oriented and functional framework for verification in Scala.
This framework will leverage Scala's conciseness with the
combination of object-oriented programming with functional programming.
Within our verification framework, we will support mixed language verification.
Verilog can easily be combined with Chisel, as Chisel generates Verilog, and
we will use ChiselTest as a driver for the open-source Verilog simulator Verlator.
With Yosys synthesis suite~\cite{Yosys} and GHDL~\cite{ghdl}
we will translate VHDL into Verilog.
The framework will also support co-verification of hardware and software

%A verification method is only usable when it can handle mixed-source designs.
%This means a Scala driven method must be able to test components written in Verilog,
%VHDL, and SystemVerilog.
%
%Chisel has support for black boxes, which allows the use of Verilog code within the Chisel design.
%Therefore, it is relatively easy to integrate Verilog components when wrapped into a black box.
%However, this forces Chisel to use Verilator instead of Treadle to run the simulation, impacting
%startup time.
%
%Chisel does not fully support VHDL. It can support VHDL using VCS, but there is no
%open-source solution available for VHDL simulation. For companies with a lot of source code written in VHDL this is a concern, as they must be able to integrate their existing IP in a Scala/Chisel based design and verification workflow.
%All major commercial simulation and synthesis tools support mixed-language designs, but no open-source tools exist that provide the same functionality.
%
%To alleviate this issue, the open-source Yosys synthesis suite \cite{Yosys} can be used. Yosys is an open-source digital hardware synthesis suite for Verilog. Yosys also has a variety of plugins, one of these being a plugin for using GHDL \cite{ghdl}, an open-source VHDL simulator. By using Yosys in conjunction with GHDL, VHDL files are compiled to an RTL-based intermediate representation, which is then written to a Verilog file using Yosys. GHDL has full support for IEEE 1076 VHDL 1987, 1993, 2002, and a subset of 2008. The workflow can be seen in Figure \ref{fig:VHDL2Verilog}. A working solution named VHDL2Verilog has been made for this, which has been tested with certain simple VHDL designs \cite{vhdl2verilog}.



%\begin{itemize}
%\item Bus functional models
%\item Multiple languages
%\item Still talk about a small example taking it through all variations
%\item WP on VHDL generation from Chisel for better verification
%\item p69: HDL models are SW projects
%\end{itemize}



%% Let us consider this one below as a bonus WP in our backyard.
%% \subsection{WP2}
%% We can then consider to generate specification automatically, \`a la
%% sanitizers used in the compiler techniques. This WP focuses on
%% low-level properties derived from the program syntax, {\em, e.g.},
%% accessing an array should be within a bound, or performing an
%% arithmetic calculation should not overflow.



%% \paragraph{MS2.} The milestone in this WP is to develop a program transformer
%% that injects in the software source, or its binary form, a list of
%% low-level specifications.

%% Related to this WP, our project collaborator Zhoulai Fu (together with
%% a hired security hacker) used sanitizers to find more than 200 bugs in
%% the Robot Operating System
%% (ROS)~\cite{web:ros-sanitizer-logs}. Besides, Zhoulai Fu has developed
%% several program transformers previously for testing floating-point
%% computation~\cite{DBLP:conf/pldi/FuS19,DBLP:conf/oopsla/FuBS15}.


%\paragraph{Learning Specification from Historical Bug Patterns.}
%We will consider generating specifications from previously known bug
%patterns.  We will collaborate with our hardware engineers to explore
%a history of hardware issues triggered by software defects and get
%patterns from which we generate specification.  Such bug patterns in
%Java, for example, can be found in \cite{web:findbugs_bugs}.  This
%step will generate specification in a syntax-driven way, relating to
%the expected functional behavior of the hardware.



%% Related to this, our collaborator Zhoulai Fu has worked on a
%% comprehensive study of developing bug finders by learning from history
%% bug patterns in the Robot Operating
%% System~\cite{nielsenFSW2020dependencybugs}


%% Let us consider this as another backyard WP.
%% \subsection{WP5} We plan to deploy our automated testing solution  in a continuous
%% integration service, presumably on the cloud, such as Jenkins or
%% Travis~\cite{DBLP:journals/tse/GallabaM20}. For each code commit, the
%% service will automatically generate the specification as implemented
%% in WP2 and WP3, and then will generate test data with a fuzzing engine
%% as in WP4.

%% \paragraph{MS5.} The milestone is to have the server established that implements
%% the workflow. It should produce a detailed status report for each
%% commit.  A similar kind of status report can be found in Google's
%% OSS-Fuzz (Continuous Fuzzing for Open Source
%% Software)~\cite{web:oss-fuzz} for example.




\section*{Value Creation}


Digital technologies are currently the main driver for growth in Denmark.
Furthermore, the Covid-19 induced working from home, gives a further boost to digitalization right now.
This project with the cooperation between researchers as DTU and ITU and
Danish companies will allow Denmark to take the lead in digital research and development.


DTU and ITU will advance the research in the design and testing of
digital systems. Our proposed approach provides a general software
engineering procedure that we plan to validate in the Robot Operating
System (closely related to co-PI's previous H2020 project on robotics) before
applying to real-world hardware systems.
This research will drive the adaption of the education curriculum towards modern tools and agile methods.



The industrial project partners will learn and get exposed to modern tools and methods to increase the efficiency of designing and testing digital systems. Our partner companies will benefit from our advanced software tool that generates test stimuli achieving high coverage of the hardware in simulation.


A postdoc and a master student, will become highly educated engineers in need in the Danish or international digital systems design industry. Those students will also learn how to program the accelerators in the cloud for future applications. This research will drive the adaption of the education curriculum towards modern tools and agile methods.


The results from the project will be available as open-source under the
industry-friendly BSD license.
%Open-source research projects attract
%other researchers, developers, and industrial partners
%to use and build on the results of the project.
A project web site will host the project documentation, the published papers, and the design's source code.
We will provide unrestricted and cost-free digital access to all research and development results.
%
We will use the developed method and tools to train a new generation of HW/SW engineers
at DTU and ITU.
In the middle of the project, we will have a coordination workshop with the project partners.
In the end, we will organize a design and verification workshop, including a hands-on tutorial,
open to all interested companies, and students from DTU and ITU.

The changes in the hardware industry indicates that the use of FPGAs will increase: A few years ago Intel bought Altera, one of the two largest FPGA production companies, to include FPGAs in future versions of their processors. Similar, AMD is aiming to buy Xilinx, the other big FPGA vendor. In addition, one can already rent a server in the cloud from Amazon that includes an FPGA. These changes all points towards that FPGAs are entering mainstream computing.

Many mainstream programming languages like C\# or Java already include functional features such as lambda expressions or higher-order functions. The more common languages for encoding FPGAs are Verilog, a C inspired language, and VHDL, a Pascal inspired language, Therefore, it may be efficient for mainstream software developers to use a functional language to efficiently implement algorithms in FPGAs and thus both increase performance and reduce the energy consumption.


\newpage
\section*{Project Plan: Key Activities and Milestones}



%The Embedded Systems Engineering section at DTU Compute provides
%the intellectual environment and the infrastructure (e.g., regression test server...) that we need for an ambitious research project.
%Furthermore, DTU Compute provides the infrastructure (e.g., an automatic test
%environment for regressions tests, web server).
%\todo{Add ITU}


\paragraph*{Key Results} The key results of the project are an verification framework for digital design, build
out of modern programming praxis in object-oriented and functional style and a program transformer
to convert bug patterns to specification in code. Both tools will be in open-source. The goal for researchers
is publication of at least 2 conference papers.

For external partners, the usage of modern development and verification tools will speedup the
throughput of development. The partners will learn new tools and methodologies from the HaaS project.
Furthermore, the external partners will benefit from a workforce of highly educated PhD
and master students in their relevant area of computer engineering and digital design.



\paragraph*{Industrial Cooperation}

Danish industry in digital design for ASICs and FPGAs is currently in transition from using traditional
test benches written in VHDL and Verilog to a verification method based on constraint random
test generation with tools such as UVM. Therefore, the HaaS project is just-in-time to support this
transition.

\paragraph*{Use Cases}

We will have several non-trivial use cases from industry and from our own development to verify
our development.
Microchip provides the specification of a hardware sorting algorithm.
From WSA we will use a decimation filter written in VHDL.
From our research we will use a multicore device, a network-on-chip~\cite{s4noc:nocarc2019},
to explore concurrent, transaction based verification.



%\vspace{-2mm}
\begin{table*}% [h!]
{\small
  \begin{center}
    \begin{tabular}{lccp{110mm}l}
      \toprule
      Task                   & PM  & Person    & Description                                                                                   \\
      \midrule
      Cover                  & 2  & postdoc      & Adding constraint random testing and coverage to ChiselTest.                     \\
      Generate               & 2   & postdoc      & Developing a hardware generator methodology by using functional programming in Scala.         \\
      Framework              & 2   & postdoc      & Development of an object-oriented and functional testing framework, including co-verification                     \\
      \midrule
      Prepare2                    & 1   & student      & Explore related work on property-based testing and ScalaCheck.                   \\
      Spec          & 2   & student      & Collaborate with hardware developers to manually inject both valid and faulty specifications. \\
      Testing & 1   & student      & Develop property-based testing the specifications with ScalaCheck.                                                      \\
      \midrule
      TeachSW                 &  1  & JM            & Make Chisel and its design methods accessible for software designers. \\
      TeachVerify                & 1   & MS        & Adding a chapter on verification to the Chisel book~\cite{chisel:book}.                     \\
      \bottomrule
    \end{tabular}
  \end{center}
  \caption{Tasks for ExHaaS}\label{tab:packages}
}
\end{table*}



The project is divided into several tasks, as shown in  Table~\ref{tab:packages}.
For each task, the time is given in person-months (PM).


For an assessment of the project's success, we plan the following milestones and deliverables:

\textbf{M1} (Month 3): Tools have been learned, and first artifacts have been developed.

\textbf{D1} (Month 3): Constraint random testing tools and valid and faulty specifications.


\textbf{M2} (Month 9): All development has been finished, and the different components
can be used for exploration and evaluation of the results with the industrial use cases.

\textbf{D2} (Month 9): Tools: testing framework and a program transformer.

\textbf{M3} (Month 12): The project has finished and two papers have been submitted to conferences.

\newpage
\small
\bibliographystyle{abbrv}
%\bibliography{myown,jsp,noc,misc,msbib}
\bibliography{../../msbib,testing,../../chisel-uvm}

\end{document}
